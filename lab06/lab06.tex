%&"../net"
\endofdump
\tikzexternalize[prefix=cache/]{lab06}
\begin{document}
    \title{Overlay Network and VXLAN}
    \maketitle
    \tableofcontents
    \vfill
    An overlay network can be thought of as a computer network on top of another network. 

    VXLAN is often described as an overlay technology because it allows to stretch Layer 2 connections over an intervening Layer 3 network by encapsulating (tunneling) Ethernet frames in a VXLAN packet that includes IP addresses.
    \vfill
    \clearpage
    \section{建立网络}

    

    \section{Wireshark 抓包}
    Use Wireshark to monitor the interfaces s1 and eth0, and describe your findings.

    \section{iperf 测试}
    Use iperf to test the network bandwidth between the two virtual machines 
    \begin{itemize}
        \item Test the bandwidth between 192.168.56.127 and 192.168.56.128
        \item Test the bandwidth between 10.0.0.1/10.0.0.2/10.0.0.101 and 10.0.0.102 (hint: you may need to specify a reasonable MTU size in order for your iperf to work in this case. Please also think about why.)
    \end{itemize}

    Compare the above results and explain the reason. 

    \section{ping 测试}
    Similar to Q2, use ping to test the network latency and analyze your results.
    
\end{document}